\section{Implementation}
In this chapter, we will present all the used hardware and software to implement the models, as well as describing all the techniques and parameters we used for training and evaluation. The implementation can be found on GitHub\footnote{\url{https://github.com/Jan-Eike/Masterarbeit}}.

\subsection{Hardware}
All evaluations were run on the same hardware setup (\ref{fig:hardware1}).

\begin{table}[H]
  \begin{center}
   	\begin{tabular}{|| c | c ||}
   	\hline
   	Hardware name & Specific model \\
   	\hline\hline
   	CPU & AMD Ryzen 7 3700X \\
 	\hline
 	GPU & NVIDIA GeForce RTX 2070 SUPER 8GB \\
 	\hline
 	RAM & 32GB \\
 	\hline
 	Mainboard & ASUS ROG STRIX B450-F Gaming \\
 	\hline
 	System & Windows 10 (64-bit) \\
 	\hline
	\end{tabular}
  \end{center}
  \caption{The hardware setup.}%
  \label{fig:hardware1}
\end{table}

\subsection{BERT and RoBERTa} \label{sec:secbert}
For the implementation of both language models, we use the Huggingface Transformers framework \cite{transformers}. In particular, for BERT we are using the bert-base-uncased model \cite{bertbaseuncased}; for RoBERTa we are using roberta-base \cite{robertabase}. The framework provides a straightforward way to fine tune and evaluate these LMs on a GPU \cite{berttraining}. We are using the same way of fine tuning and evaluation. We are fine tuning for $3$ epochs with a learning rate of $5 \cdot 10^{-5}$, $100$ warm up steps, and a batch size of $64$. For all the other parameters, we are using the standard parameters from the config.json file provided by both models. Both configurations are the same for the baseline models as well as the models used for masking. After fine tuning, we select the model with the highest f1-score achieved on the development dataset. \\
The baseline models are directly evaluated on the test set; the masked LMs are used for calculating linking words probabilities as described in chapter \ref{sec:linkingembeddings}.

\subsection{LightGBM Classifier and Feature Extraction}
To implement LightGBM, we are going to use the lightgbm python package \cite{lgbmpython}. Parameters are chosen based on the description given on the LightGBM website \cite{lgbmpython} and can be found in figure \ref{fig:lgbmparams}. We decided against extensive hyperparameter tuning since we generally did not achieve better performance with different configurations.

\begin{table}[H]
  \begin{center}
   	\begin{tabular}{|| c | c ||}
   	\hline
   	Parameter name & Parameter value \\
   	\hline\hline
   	boosting type & gbdt \\
 	\hline
 	objective & binary \\
 	\hline
 	metric & f1 \\
 	\hline
 	num leaves & 21 \\
 	\hline
 	learning rate & 0.05 \\
 	\hline
 	feature fraction & 0.9 \\
 	\hline
 	bagging fraction & 0.8 \\
 	\hline
 	bagging freq & 5 \\
 	\hline
	\end{tabular}
  \end{center}
  \caption{LightGBM Parameters.}%
  \label{fig:lgbmparams}
\end{table}

We train all LightGBM models for $10000$ epochs with early stopping patience of $40$ epochs. The model with the highest f1-score on the dev set is then used for evaluation the test set. \\
For feature extraction, we are doing a 5-fold cross validation, repeated two times on the training set, using the same parameters for the LightGBM models as for classification. Afterwards, we sort the features by average importance extracted from the LightGBM models and discard all features with an average importance of less than $\frac{\text{average importance}}{4}$. We decided to use this value since a lot of features get a very low importance, which drags down the average quite a bit, and we do not want to discard too many features.


\subsection{Neural Network Classifier}
For the neural network classifiers, we are using pytorch \cite{pytorch} for implementation, training, and evaluation. We train all classifiers for $200$ epochs with a learning rate of $2 \cdot 10^{-3}$, batch size of $32$, and an early stopping patience of $10$ epochs on a GPU. The model with the lowest loss on the dev set is then used for evaluation on the test set.

\subsection{Complete Training vs. individual Training}
In the sections before, we explained how we fine tuned the LMs and trained classifiers individually, i.e. fine tuning the LMs first, concatenating their outputs with the linking word embeddings and training a classifier on these combined embeddings afterwards. In this section, we will give a quick overview on a different approach in which we fine tuned the LMs and trained the classifiers (in this case only the neural network classifier) together. \\
For the implementation we use the same frameworks as before, but instead of training them separately, we build one model using pytorch which consists of the parts shown in Figure \ref{fig:model-architecture1} and train it in one go using the Huggingface Transformers framework. We use the same hyperparameters as for fine tuning the LMs separately (\ref{sec:secbert}).