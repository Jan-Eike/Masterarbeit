\section{Introduction}
\subsection{Motivation}
Often times in natural language, one uses certain phrases to connect sentences, especially in a discourse setting. These phrases, called discourse markers, are used to connect parts of a text or conversation \cite{Discours1999}. They are also called linking words, which is the term we are going to use over the course of this thesis. Examples of these linking words are \say{therefore}, \say{because}, \say{yet}, \say{in conclusion}, and \say{however}. The word \say{therefore} can, for example, indicate a reasoning while the phrase \say{in conclusion} indicates a conclusion. \\
In this thesis, we want to investigate whether or not linking words can help improve classifiers to identify argument structures in the English language by creating additional features based on probabilities for these linking words to be at a certain position in a text. We will determine these probabilities using masked language models.
\subsection{Tasks}
We are using two classification tasks to evaluate the performance of our approach, namely argument validity prediction and argument stance detection. The former describes the problem of determining whether a conclusion for a given text is valid. The latter is the problem of determining whether a given argument is for or against a certain topic. \\
