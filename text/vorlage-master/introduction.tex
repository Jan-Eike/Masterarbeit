\section{Introduction}
\subsection{Motivation}
Often times in natural language, one uses certain phrases to connect sentences, especially in a discourse setting. These phrases are called discourse markers and they are used to connect parts of a text or conversation \cite{Discours1999}. They are also called linking words which is the term we are going to use in the course of this thesis. Examples for these linking words are \say{therefore}, \say{because}, \say{yet}, \say{in conclusion} and \say{however}. The word \say{therefore} can for example indicate a reasoning and the phrase \say{in conclusion} indicates a conclusion. \\
In this thesis, we want to investigate whether or not linking words can help improving classifiers to identify argument structures in English language by creating additional features based on probabilities for these linking words to be at a certain position in a text. We will determine these probabilities by using masked language models.
\subsection{Task}
We are using two classification tasks to evaluate the performance of our approach, namely argument validity prediction and argument stance detection. Former describes the problem of determining whether, given a pair of some text and a conclusion, the conclusion is valid or not. Argument stance detection is the problem of classifying the stance to a certain topic, given an argument for or against it. \\
